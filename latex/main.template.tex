%%%%%%%%%%%%%%%%%
% This is an example CV created using altacv.cls (v1.1.4, 27 July 2018) written by
% LianTze Lim (liantze@gmail.com), based on the
% Cv created by BusinessInsider at http://www.businessinsider.my/a-sample-resume-for-marissa-mayer-2016-7/?r=US&IR=T
%
%% It may be distributed and/or modified under the
%% conditions of the LaTeX Project Public License, either version 1.3
%% of this license or (at your option) any later version.
%% The latest version of this license is in
%%    http://www.latex-project.org/lppl.txt
%% and version 1.3 or later is part of all distributions of LaTeX
%% version 2003/12/01 or later.
%%%%%%%%%%%%%%%%

%% Use the "normalphoto" option if you want a normal photo instead of cropped to a circle
% \documentclass[10pt,a4paper,normalphoto]{altacv}

\documentclass[10pt,a4paper,ragged2e]{altacv}
%% AltaCV uses the fontawesome5 and packages.
%% See http://texdoc.net/pkg/fontawesome5 for full list of symbols.

% Change the page layout if you need to
\geometry{left=1cm,right=1.25cm,top=1.5cm,bottom=1.5cm,columnsep=1.2cm}

% The paracol package lets you typeset columns of text in parallel
\usepackage{paracol}


% Change the font if you want to, depending on whether
% you're using pdflatex or xelatex/lualatex
\ifxetexorluatex
  % If using xelatex or lualatex:
  \setmainfont{Roboto Slab}
  \setsansfont{Lato}
  \renewcommand{\familydefault}{\sfdefault}
\else
  % If using pdflatex:
  \usepackage[rm]{roboto}
  \usepackage[defaultsans]{lato}
  % \usepackage{sourcesanspro}
  \renewcommand{\familydefault}{\sfdefault}
\fi

% Change the colours if you want to
$for(colors)$
\definecolor{$colors.name$}{HTML}{$colors.code$}
$endfor$
$for(color_scheme)$
\colorlet{$color_scheme.name$}{$color_scheme.color$}
$endfor$


% Change some fonts, if necessary
\renewcommand{\namefont}{\Huge\rmfamily\bfseries}
\renewcommand{\personalinfofont}{\footnotesize}
\renewcommand{\cvsectionfont}{\LARGE\rmfamily\bfseries}
\renewcommand{\cvsubsectionfont}{\large\bfseries}


% Change the bullets for itemize and rating marker
% for \cvskill if you want to
\renewcommand{\cvItemMarker}{{\small\textbullet}}
\renewcommand{\cvRatingMarker}{\faCircle}
% ...and the markers for the duration/location for \cvevent
% \renewcommand{\cvDateMarker}{\faCalendar*[regular]}
% \renewcommand{\cvLocationMarker}{\faMapMarker*}


%% Use (and optionally edit if necessary) this .tex if you
%% want to use an author-year reference style like APA(6)
%% for your publication list
% \input{pubs-authoryear.cfg}

%% Use (and optionally edit if necessary) this .tex if you
%% want an originally numerical reference style like IEEE
%% for your publication list
%\input{pubs-num.cfg}

%% sample.bib contains your publications
%\addbibresource{sample.bib}

\begin{document}
\name{$personal_info.name$}
\tagline{$personal_info.job_title$}
%% You can add multiple photos on the left or right
\photoR{2.5cm}{img.jpg}
% \photoL{2.5cm}{Yacht_High,Suitcase_High}

\personalinfo{%
  % Not all of these are required!
  % You can add your own with \printinfo{symbol}{detail}
  \email{$personal_info.name$}
  \phone{$for(personal_info.phone)$$personal_info.phone$$sep$ / $endfor$}
  \location{$personal_info.location$}
  \linkedin{$personal_info.linkedin$}
  \github{$personal_info.github$}
}

%% Make the header extend all the way to the right, if you want.
%\begin{fullwidth}
\makecvheader
%\end{fullwidth}

%% Depending on your tastes, you may want to make fonts of itemize environments slightly smaller
\AtBeginEnvironment{itemize}{\small}

%% Set the left/right column width ratio to 6:4.
\columnratio{0.6}

% Start a 2-column paracol. Both the left and right columns will automatically
% break across pages if things get too long.
\begin{paracol}{2}
\cvsection{Experience}
$for(experiences)$
\cvevent{$experiences.role$}{$experiences.company$}{$experiences.duration$}{$experiences.location$}
\begin{itemize}
$for(experiences.responsibilities)$
  \item $experiences.responsibilities$
$endfor$
\end{itemize}
$endfor$

$if(hobbies)$
\cvsection{Hobbies}
$for(hobbies)$
\cvachievement{$hobbies.icon$}{$hobbies.title$}{$hobbies.description$}
$endfor$
$endif$

$if(project)$
\cvsection{Projects}
$for(projects)$
\cvachievement{$projects.icon$}{$projects.title$}{$projects.description$}
\divider
$endfor$
$endif$

\switchcolumn

\cvsection{Life Philosophy}
\begin{quote}
'' You are the only person on earth who can use your ability ''
\end{quote}
\cvsection{Education}
\cvevent{B.Tech \ in Computer Science } {Dharmsinh Desai University}{Jul 2013 -- May 2016} {Nadiad , Gujarat}
\cvevent{Diploma \ in Computer Science } {Gujarat Technical University}{Jul 2010 -- May 2013} {Navsari , Gujarat}

\cvsection{Technical Strength}
\cvskill{AngularJS 2+}{4}
% \divider
\cvskill{ReactJS/React Native}{3}
% \divider
\cvskill{Javascript }{4}
% \divider
\cvskill{ Node JS }{3}
% \divider
\cvskill{BackBone.JS }{3}
% \divider
\cvskill{MongoDB }{3}
% \divider
\cvskill{iONIC Framework}{4}
% \divider
\cvskill{Cross platform Era}{4}
% \divider
\cvskill{Android}{3}
% \divider
\cvskill{.Net MVC Technology}{3}

\cvsection{Most Proud of}
\cvachievement{\faTrophy}{Best Mobile App Developer }{Individual working on various mobile apps. and sink to deliver before time}

\divider
\cvachievement{\faLineChart}{Deliver application on Time}{ Continuously work on various platform and expand learning capability in terms of quality }

\cvsection{Strengths}
\cvtag{Hard-working (18/24)} 
\cvtag{Persuasive}
\cvtag{Motivator \& Leader}
\cvtag{UX}
\cvtag{Mobile Devices \& Applications}
\cvtag{ Product Management\& Logic Designer}

\cvsection{Languages}
\cvskill{English}{4}
% \divider
\cvskill{Hindi}{4}
% \divider
\cvskill{Marathi}{3}
% \divider
\cvskill{Gujarati}{4}






\pagebreak


\cvsection{Apps. on Store}

\cvevent{Gold Guide of Dragon City Gems} {Dharmsinh Desai University}{} {}
\cvevent{Hetrix} {Dharmsinh Desai University}{}{}
\cvevent{Parity Game } {Gujarat Technical University}{} {}


\cvsection{Referees}

% \cvref{name}{email}{mailing address}
\cvref{HOD.\ CK Bhensadiya}{ckbhensadiya@gmail.com}
{Dharmsinh Desai University}

\divider

\cvref{Project Manager.\ Shishir Mishra}{shishir.bobby@gmail.com}
{Synoverge Technology}



\end{paracol}



\end{document}
