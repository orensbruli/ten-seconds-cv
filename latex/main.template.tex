%%%%%%%%%%%%%%%%%
% Highly inspired in the sample CV template created using altacv.cls
% (v1.6.5, 3 Nov 2022) written by LianTze Lim (liantze@gmail.com). Compiles with pdfLaTeX, XeLaTeX and LuaLaTeX.
%
%% It may be distributed and/or modified under the
%% conditions of the LaTeX Project Public License, either version 1.3
%% of this license or (at your option) any later version.
%% The latest version of this license is in
%%    http://www.latex-project.org/lppl.txt
%% and version 1.3 or later is part of all distributions of LaTeX
%% version 2003/12/01 or later.
%%%%%%%%%%%%%%%%

%% Use the "normalphoto" option if you want a normal photo instead of cropped to a circle
% \documentclass[10pt,a4paper,normalphoto]{altacv}

\documentclass[10pt,a4paper,ragged2e,withhyper]{altacv}
%% AltaCV uses the fontawesome5 and packages.
%% See http://texdoc.net/pkg/fontawesome5 for full list of symbols.

% Change the page layout if you need to
\geometry{left=1cm,right=1.25cm,top=1cm,bottom=1.5cm,columnsep=1.2cm}

% The paracol package lets you typeset columns of text in parallel
\usepackage{paracol}
\usepackage{adjustbox}


% Change the font if you want to, depending on whether
% you're using pdflatex or xelatex/lualatex
\ifxetexorluatex
  % If using xelatex or lualatex:
  \setmainfont{Roboto Slab}
  \setsansfont{Lato}
  \renewcommand{\familydefault}{\sfdefault}
\else
  % If using pdflatex:
  \usepackage[rm]{roboto}
  \usepackage[defaultsans]{lato}
  % \usepackage{sourcesanspro}
  \renewcommand{\familydefault}{\sfdefault}
\fi

% Change the colours if you want to
$for(colors)$
\definecolor{$colors.name$}{HTML}{$colors.code$}
$endfor$
$for(color_scheme)$
\colorlet{$color_scheme.name$}{$color_scheme.color$}
$endfor$


% Change some fonts, if necessary
\renewcommand{\namefont}{\Huge\rmfamily\bfseries}
\renewcommand{\personalinfofont}{\footnotesize}
\renewcommand{\cvsectionfont}{\LARGE\rmfamily\bfseries}
\renewcommand{\cvsubsectionfont}{\large\bfseries}


% Change the bullets for itemize and rating marker
% for \cvskill if you want to
\renewcommand{\cvItemMarker}{{\small\textbullet}}
\renewcommand{\cvRatingMarker}{\faCircle}
% ...and the markers for the duration/location for \cvevent
% \renewcommand{\cvDateMarker}{\faCalendar*[regular]}
% \renewcommand{\cvLocationMarker}{\faMapMarker*}


%% Use (and optionally edit if necessary) this .tex if you
%% want to use an author-year reference style like APA(6)
%% for your publication list
% \input{pubs-authoryear.cfg}

%% Use (and optionally edit if necessary) this .tex if you
%% want an originally numerical reference style like IEEE
%% for your publication list
%\input{pubs-num.cfg}

%% sample.bib contains your publications
%\addbibresource{sample.bib}

\begin{document}
\name{$personal_info.name$}
\tagline{$personal_info.job_title$}
%% You can add multiple photos on the left or right
\photoR{2.5cm}{img.jpg}
% \photoL{2.5cm}{Yacht_High,Suitcase_High}

\personalinfo{%
  % Not all of these are required!
  % You can add your own with \printinfo{symbol}{detail}
  \email{$personal_info.name$}
  \phone{$for(personal_info.phone)$$personal_info.phone$$sep$ / $endfor$}
  \location{$personal_info.location$}
  \linkedin{$personal_info.linkedin$}
  \github{$personal_info.github$}
}

%% Make the header extend all the way to the right, if you want.
%\begin{fullwidth}
\makecvheader
%\end{fullwidth}

%% Depending on your tastes, you may want to make fonts of itemize environments slightly smaller
\AtBeginEnvironment{itemize}{\small}

%% Set the left/right column width ratio to 6:4.
\columnratio{0.6}

% Start a 2-column paracol. Both the left and right columns will automatically
% break across pages if things get too long.
\begin{paracol}{2}
\cvsection{Experience}
$for(experiences)$
\cvevent{$experiences.role$}{$experiences.company$}{$experiences.duration$}{$experiences.location$}
{$experiences.responsibilities$}

\divider
$endfor$

\newpage

$if(hobbies)$
\cvsection{Hobbies}
$for(hobbies)$
\cvachievement{$hobbies.icon$}{$hobbies.title$}{$hobbies.description$}
$endfor$
$endif$

$if(project)$
\cvsection{Projects}
$for(projects)$
\cvachievement{$projects.icon$}{$projects.title$}{$projects.description$}
\divider
$endfor$
$endif$



\switchcolumn
\input{sidebar}
\end{paracol}

  % Add heatmap.png image full width
\begin{figure}
  \centering
  \begin{adjustbox}{width=1.1\textwidth,center}
    \includegraphics{heatmap.png}
  \end{adjustbox}
  \caption{Tu descripción de la imagen}
  \label{fig:heatmap}
\end{figure}

\end{document}